%% LyX 2.0.5 created this file.  For more info, see http://www.lyx.org/.
%% Do not edit unless you really know what you are doing.
\documentclass[ngerman]{beamer}
\usepackage{mathptmx}
\usepackage[T1]{fontenc}
\usepackage[latin9]{inputenc}
\usepackage{amsmath}
\usepackage{amssymb}

\makeatletter
%%%%%%%%%%%%%%%%%%%%%%%%%%%%%% Textclass specific LaTeX commands.
 % this default might be overridden by plain title style
 \newcommand\makebeamertitle{\frame{\maketitle}}%
 \AtBeginDocument{
   \let\origtableofcontents=\tableofcontents
   \def\tableofcontents{\@ifnextchar[{\origtableofcontents}{\gobbletableofcontents}}
   \def\gobbletableofcontents#1{\origtableofcontents}
 }
 \long\def\lyxframe#1{\@lyxframe#1\@lyxframestop}%
 \def\@lyxframe{\@ifnextchar<{\@@lyxframe}{\@@lyxframe<*>}}%
 \def\@@lyxframe<#1>{\@ifnextchar[{\@@@lyxframe<#1>}{\@@@lyxframe<#1>[]}}
 \def\@@@lyxframe<#1>[{\@ifnextchar<{\@@@@@lyxframe<#1>[}{\@@@@lyxframe<#1>[<*>][}}
 \def\@@@@@lyxframe<#1>[#2]{\@ifnextchar[{\@@@@lyxframe<#1>[#2]}{\@@@@lyxframe<#1>[#2][]}}
 \long\def\@@@@lyxframe<#1>[#2][#3]#4\@lyxframestop#5\lyxframeend{%
   \frame<#1>[#2][#3]{\frametitle{#4}#5}}
 \newenvironment{topcolumns}{\begin{columns}[t]}{\end{columns}}
 \def\lyxframeend{} % In case there is a superfluous frame end

%%%%%%%%%%%%%%%%%%%%%%%%%%%%%% User specified LaTeX commands.
\usetheme{Warsaw}
% oder ...

\setbeamercovered{transparent}
% oder auch nicht

\makeatother

\usepackage{babel}
\begin{document}





\title[Klein, sch�n, schnell]{Klein, sch�n, schnell}


\subtitle{mit Curses den eigenen Scripten eine sch�ne Oberfl�che verleihen}


\author[]{Alexander Kluth}


\institute[]{}


\date[gpw2013]{15. deutscher Perl-Workshop 2013}

\makebeamertitle


%\pgfdeclareimage[height=0.5cm]{institution-logo}{institution-logo-filename}

%\logo{\pgfuseimage{institution-logo}}



\AtBeginSubsection[]{

  \frame<beamer>{ 

    \frametitle{Gliederung}   

    \tableofcontents[currentsection,currentsubsection] 

  }

}



%\beamerdefaultoverlayspecification{<+->}


\lyxframeend{}\lyxframe{Gliederung}

\tableofcontents{}




\lyxframeend{}\section{Einf�hrung}


\lyxframeend{}\subsection{Das Ausgangsproblem}


\lyxframeend{}\lyxframe{�berschriften sollten informativ sein\\
Korrekte Gro�-/Kleinschreibung beachten}


\framesubtitle{Untertitel sind optional}
\begin{itemize}
\item H�ufig Auflistungen benutzen.


\pause{}

\item Sehr kurze S�tze oder Phrasen verwenden.


\pause{}

\item Diese Overlays werden mit dem Absatzstil >>Pause<< erzeugt.
\end{itemize}

\lyxframeend{}\lyxframe{�berschriften m�ssen informativ sein}
\begin{itemize}
\item <1->Man kann auch Overlay-Spezifikationen benutzen, um Overlays zu
erzeugen.
\item <3->Hiermit k�nnen Punkte in beliebiger Reihenfolge pr�sentiert werden.
\item <2->Dies wird als Zweites gezeigt.
\end{itemize}

\lyxframeend{}\lyxframe{�berschriften m�ssen informativ sein}
\begin{block}
<1->{}
\begin{itemize}
\item Block ohne �berschrift.
\item Wird auf allen Overlays angezeigt.
\end{itemize}
\end{block}
\begin{exampleblock}
<2->{Ein Beispielblocktitel}
\begin{itemize}
\item $e^{i\pi}=-1$.
\item $e^{i\pi/2}=i$.
\end{itemize}
\end{exampleblock}

\lyxframeend{}\subsection{Fr�here Arbeiten}


\lyxframeend{}\lyxframe{�berschriften m�ssen informativ sein}
\begin{example}%{}
<1->Auf dem ersten Overlay.
\end{example}%{}

\begin{example}%{}
<2->Auf dem zweiten Overlay.
\end{example}%{}

\lyxframeend{}\section{Unsere Resultate/Beitrag}


\lyxframeend{}\subsection{Hauptresultate}


\lyxframeend{}\lyxframe{�berschriften m�ssen informativ sein}
\begin{theorem}%{}
Auf dem ersten Overlay.
\end{theorem}%{}

\pause{}
\begin{corollary}%{}
Auf dem zweiten Overlay.
\end{corollary}%{}

\lyxframeend{}\lyxframe{�berschriften m�ssen informativ sein}
\begin{topcolumns}%{}


\column{5cm}
\begin{theorem}%{}
<1->In der linken Spalte.
\end{theorem}%{}

\column{5cm}
\begin{corollary}%{}
<2->In der rechten Spalte.\\
Neue Zeile.
\end{corollary}%{}
\end{topcolumns}%{}

\lyxframeend{}\subsection{Ideen f�r Beweise/Umsetzung}


\lyxframeend{}\section*{Zusammenfassung}


\lyxframeend{}\lyxframe{Zusammenfassung}


\begin{itemize}
\item Die \textcolor{red}{erste Hauptbotschaft} des Vortrags in ein bis
zwei Zeilen.
\item Die \textcolor{red}{zweite Hauptbotschaft} des Vortrags in ein bis
zwei Zeilen.
\item Eventuell noch eine \textcolor{red}{dritte Botschaft}, aber das reicht
dann auch.
\end{itemize}


\vskip0pt plus.5fill
\begin{itemize}
\item Ausblick

\begin{itemize}
\item Etwas, was wir noch nicht l�sen konnten.
\item Noch etwas, das wir noch nicht l�sen konnten.
\end{itemize}
\end{itemize}

\lyxframeend{}

\appendix

\lyxframeend{}\section*{Anhang}


\lyxframeend{}\subsection*{Weiterf�hrende Literatur}


\lyxframeend{}\lyxframe{[allowframebreaks]Weiterf�hrende Literatur}

\beamertemplatebookbibitems
\begin{thebibliography}{Literaturverzeichnis}
\bibitem{Autor1990}A. Autor. \newblock \emph{Einf�hrung in das Pr�sentationswesen}.
\newblock Klein-Verlag, 1990.\beamertemplatearticlebibitems

\bibitem{Jemand2002}S. Jemand.\newblock On this and that\emph{.}
\newblock\emph{Journal on This and That}. 2(1):50--100, 2000.

\end{thebibliography}

\lyxframeend{}
\end{document}
